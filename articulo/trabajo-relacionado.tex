\section{\label{sec:trabajo-relacionado} Related work}
Rad, B. et al. \cite{rad2017introduction} presents a performance comparison between hypervisor-based and container-based technologies. 
Three scenarios are under test: Docker vs KVM, LXC vs Xen and bare metal, Docker and KVM. 
All scenarios show how container technologies exhibit a better performance when they are compared with hypervisor ones. 
Experiments with Docker also show how its performance is close to the bare metal environment.
Those experiments were not homogeneous. 
When Docker was compared with KVM, an image processing test was carried out. 
The LXC and Xen scenario run SQL queries.
And bare metal, Docker and KVM; employed reading and writing I/O operations.
The heterogeneity of the tests makes it difficult to state a clear picture about an overall performance of the different technologies under study.
However, Docker showed a good overall performance.

Torrez et. al. \cite{torrez2019hpc} studied three different HPC oriented container technologies: Charliecloud, Shifter and Singularity.
These technologies were evaluated against industry-standard benchmarks (SysBench,  STREAM,  and  HPCG).
These benchmarks are written in a wide range of programming languages such as: C, Python, Go, shell scripts amongst others.
Experimental results show little performance degradation. 
However, 1.8\% of memory degradation was found which authors stated is negligible. 
They encourage to containerize applications because of the low impact exhibited by the technologies under study.
Unfortunately, their results do not consider startup and teardown overhead. 
This overhead is not negligible when many container instances are fired.
