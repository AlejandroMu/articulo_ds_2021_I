\documentclass[conference]{IEEEtran}
%\usepackage[spanish]{babel}
% http://nokyotsu.com/latex/acentos.html El paquete inputenc permite incluir
% los caracteres con tilde. En el enlace se dan mas detalles al respecto.
\usepackage[utf8]{inputenc}
% *** MISC UTILITY PACKAGES ***
%
%\usepackage{ifpdf}
% Heiko Oberdiek's ifpdf.sty is very useful if you need conditional
% compilation based on whether the output is pdf or dvi.
% usage:
% \ifpdf
%   % pdf code
% \else
%   % dvi code
% \fi
% The latest version of ifpdf.sty can be obtained from:
% http://www.ctan.org/pkg/ifpdf
% Also, note that IEEEtran.cls V1.7 and later provides a builtin
% \ifCLASSINFOpdf conditional that works the same way.
% When switching from latex to pdflatex and vice-versa, the compiler may
% have to be run twice to clear warning/error messages.






% *** CITATION PACKAGES ***
%
%\usepackage{cite}
% cite.sty was written by Donald Arseneau
% V1.6 and later of IEEEtran pre-defines the format of the cite.sty package
% \cite{} output to follow that of the IEEE. Loading the cite package will
% result in citation numbers being automatically sorted and properly
% "compressed/ranged". e.g., [1], [9], [2], [7], [5], [6] without using
% cite.sty will become [1], [2], [5]--[7], [9] using cite.sty. cite.sty's
% \cite will automatically add leading space, if needed. Use cite.sty's
% noadjust option (cite.sty V3.8 and later) if you want to turn this off
% such as if a citation ever needs to be enclosed in parenthesis.
% cite.sty is already installed on most LaTeX systems. Be sure and use
% version 5.0 (2009-03-20) and later if using hyperref.sty.
% The latest version can be obtained at:
% http://www.ctan.org/pkg/cite
% The documentation is contained in the cite.sty file itself.






% *** GRAPHICS RELATED PACKAGES ***
%
\ifCLASSINFOpdf
  % \usepackage[pdftex]{graphicx}
  % declare the path(s) where your graphic files are
  % \graphicspath{{../pdf/}{../jpeg/}}
  % and their extensions so you won't have to specify these with
  % every instance of \includegraphics
  % \DeclareGraphicsExtensions{.pdf,.jpeg,.png}
\else
  % or other class option (dvipsone, dvipdf, if not using dvips). graphicx
  % will default to the driver specified in the system graphics.cfg if no
  % driver is specified.
  % \usepackage[dvips]{graphicx}
  % declare the path(s) where your graphic files are
  % \graphicspath{{../eps/}}
  % and their extensions so you won't have to specify these with
  % every instance of \includegraphics
  % \DeclareGraphicsExtensions{.eps}
\fi
% graphicx was written by David Carlisle and Sebastian Rahtz. It is
% required if you want graphics, photos, etc. graphicx.sty is already
% installed on most LaTeX systems. The latest version and documentation
% can be obtained at: 
% http://www.ctan.org/pkg/graphicx
% Another good source of documentation is "Using Imported Graphics in
% LaTeX2e" by Keith Reckdahl which can be found at:
% http://www.ctan.org/pkg/epslatex
%
% latex, and pdflatex in dvi mode, support graphics in encapsulated
% postscript (.eps) format. pdflatex in pdf mode supports graphics
% in .pdf, .jpeg, .png and .mps (metapost) formats. Users should ensure
% that all non-photo figures use a vector format (.eps, .pdf, .mps) and
% not a bitmapped formats (.jpeg, .png). The IEEE frowns on bitmapped formats
% which can result in "jaggedy"/blurry rendering of lines and letters as
% well as large increases in file sizes.
%
% You can find documentation about the pdfTeX application at:
% http://www.tug.org/applications/pdftex





% *** MATH PACKAGES ***
%
%\usepackage{amsmath}
% A popular package from the American Mathematical Society that provides
% many useful and powerful commands for dealing with mathematics.
%
% Note that the amsmath package sets \interdisplaylinepenalty to 10000
% thus preventing page breaks from occurring within multiline equations. Use:
%\interdisplaylinepenalty=2500
% after loading amsmath to restore such page breaks as IEEEtran.cls normally
% does. amsmath.sty is already installed on most LaTeX systems. The latest
% version and documentation can be obtained at:
% http://www.ctan.org/pkg/amsmath





% *** SPECIALIZED LIST PACKAGES ***
%
%\usepackage{algorithmic}
% algorithmic.sty was written by Peter Williams and Rogerio Brito.
% This package provides an algorithmic environment fo describing algorithms.
% You can use the algorithmic environment in-text or within a figure
% environment to provide for a floating algorithm. Do NOT use the algorithm
% floating environment provided by algorithm.sty (by the same authors) or
% algorithm2e.sty (by Christophe Fiorio) as the IEEE does not use dedicated
% algorithm float types and packages that provide these will not provide
% correct IEEE style captions. The latest version and documentation of
% algorithmic.sty can be obtained at:
% http://www.ctan.org/pkg/algorithms
% Also of interest may be the (relatively newer and more customizable)
% algorithmicx.sty package by Szasz Janos:
% http://www.ctan.org/pkg/algorithmicx




% *** ALIGNMENT PACKAGES ***
%
%\usepackage{array}
% Frank Mittelbach's and David Carlisle's array.sty patches and improves
% the standard LaTeX2e array and tabular environments to provide better
% appearance and additional user controls. As the default LaTeX2e table
% generation code is lacking to the point of almost being broken with
% respect to the quality of the end results, all users are strongly
% advised to use an enhanced (at the very least that provided by array.sty)
% set of table tools. array.sty is already installed on most systems. The
% latest version and documentation can be obtained at:
% http://www.ctan.org/pkg/array


% IEEEtran contains the IEEEeqnarray family of commands that can be used to
% generate multiline equations as well as matrices, tables, etc., of high
% quality.




% *** SUBFIGURE PACKAGES ***
%\ifCLASSOPTIONcompsoc
%  \usepackage[caption=false,font=normalsize,labelfont=sf,textfont=sf]{subfig}
%\else
%  \usepackage[caption=false,font=footnotesize]{subfig}
%\fi
% subfig.sty, written by Steven Douglas Cochran, is the modern replacement
% for subfigure.sty, the latter of which is no longer maintained and is
% incompatible with some LaTeX packages including fixltx2e. However,
% subfig.sty requires and automatically loads Axel Sommerfeldt's caption.sty
% which will override IEEEtran.cls' handling of captions and this will result
% in non-IEEE style figure/table captions. To prevent this problem, be sure
% and invoke subfig.sty's "caption=false" package option (available since
% subfig.sty version 1.3, 2005/06/28) as this is will preserve IEEEtran.cls
% handling of captions.
% Note that the Computer Society format requires a larger sans serif font
% than the serif footnote size font used in traditional IEEE formatting
% and thus the need to invoke different subfig.sty package options depending
% on whether compsoc mode has been enabled.
%
% The latest version and documentation of subfig.sty can be obtained at:
% http://www.ctan.org/pkg/subfig




% *** FLOAT PACKAGES ***
%
%\usepackage{fixltx2e}
% fixltx2e, the successor to the earlier fix2col.sty, was written by
% Frank Mittelbach and David Carlisle. This package corrects a few problems
% in the LaTeX2e kernel, the most notable of which is that in current
% LaTeX2e releases, the ordering of single and double column floats is not
% guaranteed to be preserved. Thus, an unpatched LaTeX2e can allow a
% single column figure to be placed prior to an earlier double column
% figure.
% Be aware that LaTeX2e kernels dated 2015 and later have fixltx2e.sty's
% corrections already built into the system in which case a warning will
% be issued if an attempt is made to load fixltx2e.sty as it is no longer
% needed.
% The latest version and documentation can be found at:
% http://www.ctan.org/pkg/fixltx2e


%\usepackage{stfloats}
% stfloats.sty was written by Sigitas Tolusis. This package gives LaTeX2e
% the ability to do double column floats at the bottom of the page as well
% as the top. (e.g., "\begin{figure*}[!b]" is not normally possible in
% LaTeX2e). It also provides a command:
%\fnbelowfloat
% to enable the placement of footnotes below bottom floats (the standard
% LaTeX2e kernel puts them above bottom floats). This is an invasive package
% which rewrites many portions of the LaTeX2e float routines. It may not work
% with other packages that modify the LaTeX2e float routines. The latest
% version and documentation can be obtained at:
% http://www.ctan.org/pkg/stfloats
% Do not use the stfloats baselinefloat ability as the IEEE does not allow
% \baselineskip to stretch. Authors submitting work to the IEEE should note
% that the IEEE rarely uses double column equations and that authors should try
% to avoid such use. Do not be tempted to use the cuted.sty or midfloat.sty
% packages (also by Sigitas Tolusis) as the IEEE does not format its papers in
% such ways.
% Do not attempt to use stfloats with fixltx2e as they are incompatible.
% Instead, use Morten Hogholm'a dblfloatfix which combines the features
% of both fixltx2e and stfloats:
%
% \usepackage{dblfloatfix}
% The latest version can be found at:
% http://www.ctan.org/pkg/dblfloatfix




% *** PDF, URL AND HYPERLINK PACKAGES ***
%
%\usepackage{url}
% url.sty was written by Donald Arseneau. It provides better support for
% handling and breaking URLs. url.sty is already installed on most LaTeX
% systems. The latest version and documentation can be obtained at:
% http://www.ctan.org/pkg/url
% Basically, \url{my_url_here}.




% *** Do not adjust lengths that control margins, column widths, etc. ***
% *** Do not use packages that alter fonts (such as pslatex).         ***
% There should be no need to do such things with IEEEtran.cls V1.6 and later.
% (Unless specifically asked to do so by the journal or conference you plan
% to submit to, of course. )



\begin{document}
%
% paper title
% Titles are generally capitalized except for words such as a, an, and, as,
% at, but, by, for, in, nor, of, on, or, the, to and up, which are usually
% not capitalized unless they are the first or last word of the title.
% Linebreaks \\ can be used within to get better formatting as desired.
% Do not put math or special symbols in the title.
\title{An updated performance evaluation of container-based virtualization technologies}


% author names and affiliations
% use a multiple column layout for up to three different
% affiliations
\author{
\IEEEauthorblockN{Maria Aguiar}
\IEEEauthorblockA{Universidad del Valle\\
Email: maria.aguiar@correounivalle.edu.co}
\and
\IEEEauthorblockN{Cristian Ballesteros}
\IEEEauthorblockA{Universidad del Valle\\
Email: cristian.ballesteros@correounivalle.edu.co}
\and
\IEEEauthorblockN{Alejandro Bravo}
\IEEEauthorblockA{Universidad ICESI\\
Email: 1061816906@icesi.edu.co}
\and
\IEEEauthorblockN{Santiago García}
\IEEEauthorblockA{Universidad del Valle\\
Email: xxxxx.xxxxxxxxxx@correounivalle.edu.co}
\and
\IEEEauthorblockN{Johan Mera}
\IEEEauthorblockA{Universidad del Valle\\
Email: johan.millan@correounivalle.edu.co}
\and
\IEEEauthorblockN{Orlando Montenegro}
\IEEEauthorblockA{Universidad del Valle\\
Email: xxxxx.xxxxxxxxxx@correounivalle.edu.co}
\and
\IEEEauthorblockN{Kenneth Puliche}
\IEEEauthorblockA{Universidad del Valle\\
Email: kenneth.puliche@correounivalle.edu.co}
\and
\IEEEauthorblockN{Pavel Zambrano}
\IEEEauthorblockA{Universidad del Valle\\
Email: xxxxx.xxxxxxxxxx@correounivalle.edu.co}
\and
\IEEEauthorblockN{John Sanabria}
\IEEEauthorblockA{Universidad del Valle\\
Email: john.sanabria@correounivalle.edu.co}
}

\maketitle

\begin{abstract}
La virtualización basada en contenedores ha logrado consolidarse en los entornos académicos y de la industria gracias al rendimiento y su alta densidad de ambientes virtuales en un ambiente anfitrión. 
El ecosistema de contenedores y su bajo \emph{footprint} han facilitado la distribución de ambientes virtuales de contenedores favoreciendo la reproducibilidad no solo de entornos productivos sino también de ambientes de pruebas.
Gracias a estos factores hay cada vez más entornos que adoptan estas tecnologías en sus líneas de producción y gestión de información. 
Sin embargo, los cambios y las mejoras se hacen necesarios.
Se identifican posibles vulnerabilades y oportunidades de mejoran que favorecen la adopción de cambias y nuevas características y la aparición de nuevas alternativas de virtualización en este contexto. 
Así mismo, el kernel del sistema operativo se modifica y esto afecta el rendimiento de los entornos virtualizados.
Este artículo busca consolidar la evaluación de tres de las herramientas de virtualización más populares como son: LXC, Docker y CoreOS. 
Estas herramientas son puestas a prueba a la hora de acceder a la CPU, memoria RAM, disco y red. 
Así mismo, se evalúa el rendimiento de un par de motores de bases de datos, uno en el área de las bases de datos relacionales y  otro de las bases de datos NoSQL, sobre los tres entornos de virtualización descritos anteriormente.
\end{abstract}

\section{Introducción}
La sección de introducción.


\section{\label{sec:trabajo-relacionado} Related work}
Rad, B. et al. \cite{rad2017introduction} presents a performance comparison between hypervisor-based and container-based technologies. 
Three scenarios are under test: Docker vs KVM, LXC vs Xen and bare metal, Docker and KVM. 
All scenarios show how container technologies exhibit a better performance when they are compared with hypervisor ones. 
Experiments with Docker also show how its performance is close to the bare metal environment.
Those experiments were not homogeneous. 
When Docker was compared with KVM, an image processing test was carried out. 
The LXC and Xen scenario run SQL queries.
And bare metal, Docker and KVM; employed reading and writing I/O operations.
The heterogeneity of the tests makes it difficult to state a clear picture about an overall performance of the different technologies under study.
However, Docker showed a good overall performance.

Torrez et. al. \cite{torrez2019hpc} studied three different HPC oriented container technologies: Charliecloud, Shifter and Singularity.
These technologies were evaluated against industry-standard benchmarks (SysBench,  STREAM,  and  HPCG).
These benchmarks are written in a wide range of programming languages such as: C, Python, Go, shell scripts amongst others.
Experimental results show little performance degradation. 
However, 1.8\% of memory degradation was found which authors stated is negligible. 
They encourage to containerize applications because of the low impact exhibited by the technologies under study.
Unfortunately, their results do not consider startup and teardown overhead. 
This overhead is not negligible when many container instances are fired.


\section{\label{sec:intro-despliegue}Despliegue de tecnologías de virtualización}

En la presente sección se lleva a cabo la descripción de los procesos de despliegue de las herramientas Docker (sección \ref{sec:despliegue-docker}), LXC (sección\ref{sec:despliegue-lxc}) y CoreOS (sección \ref{sec:despliegue-coreos}). 

  \subsection{\label{sec:despliegue-docker}Docker}

In order to install Docker, the following steps were performed:

\begin{verbatim}
$ curl -fsSL https://get.docker.com -o get-docker.sh
$ sh get-docker.sh
\end{verbatim}

Once \emph{Docker} is installed, \emph{MySql} and \emph{PostgreSQL} images were deployed by applying the following steps:

\textbf{MySql}:

\begin{verbatim}
# Pull the MySQL Docker Image
$ docker pull mysql/mysql-server:8.0

# Deploy the MySQL Container
$ docker run --name=my-sql-docker -d mysql/mysql-server:8.0

# Start a MySQL client inside the container
$ docker exec -it my-sql-docker mysql -uroot -p
\end{verbatim}

\textbf{PostgreSQL}:

\begin{verbatim}
# Pull the Postgres Docker Image
$ docker pull postgres:13.2

# Deploy the Postgres Container
$ docker run --name posCont -e POSTGRES_PASSWORD=pwd -p 5432:5432 -d postgres:13.2

# Start a Postgres client inside the container
$ docker exec -it posCont psql -U postgres
\end{verbatim}

  \subsection{\label{sec:despliegue-lxc}LXC}

Para llevar a cabo el despliegue de LXC se puede hacer a través del comando:


  \subsection{\label{sec:despliegue-coreos}CoreOS}
En esta sección se presenta la forma como se lleva a cabo el despliegue de CoreOS.

  \subsection{\label{sec:baremetal} \emph{Bare metal}}


\section{\label{sec:benchmarking} Herramientas de evaluación de desempeño}


\section{\label{sec:entorno-pruebas} Test environment}
% Es importante considerar aquí lo relativo al kernel bajo el cual se está haciendo la evaluación.
% Es importante validar la versión del kernel que usa CoreOS y tratar de ser lo más homogéneo respecto a que Docker, LXC y CoreOS usen la misma versión de kernel.


\section{\label{sec:resultados} Resultados}


\section{\label{sec:trabajo-futuro} Future work}


\section{\label{sec:conclusiones} Conclusiones}
La sección de conclusiones aquí.

% references section

% can use a bibliography generated by BibTeX as a .bbl file
% BibTeX documentation can be easily obtained at:
% http://mirror.ctan.org/biblio/bibtex/contrib/doc/
% The IEEEtran BibTeX style support page is at:
% http://www.michaelshell.org/tex/ieeetran/bibtex/
%\bibliographystyle{IEEEtran}
% argument is your BibTeX string definitions and bibliography database(s)
%\bibliography{IEEEabrv,../bib/paper}
%
% <OR> manually copy in the resultant .bbl file
% set second argument of \begin to the number of references
% (used to reserve space for the reference number labels box)
\begin{thebibliography}{1}

\bibitem{rad2017introduction}
Rad, B., Bhatti, H. \& Ahmadi, M. An introduction to docker and analysis of its performance.  {\em International Journal Of Computer Science And Network Security (ijcsns)}.  \textbf{17}, 228 (2017).

\bibitem{torrez2019hpc}
Torrez, A., Les, T. \& Priedhorsky, R. HPC container runtimes have minimal or no performance impact. {\em 2019 IEEE/ACM International Workshop on Containers and New Orchestration Paradigms for Isolated Environments in HPC (CANOPIE-HPC)}. \textbf{37--42}, (2019).

\bibitem{saransig2018performance}
Saransig, A. \& Tapia, F. Performance analysis of monolithic and micro service architectures--containers technology. {\em International Conference on Software Process Improvement}. \textbf{270--279}, (2018).

\bibitem{ruiz2015performance}
Ruiz, A. \& Jeanvoine, E. \& Nussbaum, L. Performance evaluation of containers for HPC. {\em European Conference on Parallel Processing}. \textbf{813--824}, (2015).


\bibitem{soltesz2007container}
Soltesz, S., P{\"o}tzl, H., Fiuczynski, M., Bavier, A. \& Peterson, L. Container-based operating system virtualization: a scalable, high-performance alternative to hypervisors. {\em Proceedings of the 2Nd ACM SIGOPS/EuroSys european conference on computer systems 2007}, \textbf{275--287}, (2007).


\bibitem{chae2019performance}
Chae, M., Lee, H. \& Lee, K. A performance comparison of linux containers and virtual machines using Docker and KVM.  {\em Cluster Computing}. 	extbf{22}, 1765--1775 (2019).


\bibitem{casalicchio2017measuring}
Casalicchio, E. \& Perciballi, V. Measuring docker performance: What a mess!!!. {\em Proceedings of the 8th ACM/SPEC on International Conference on Performance Engineering Companion}, \textbf{11--16} (2017).

\bibitem{kozhirbayev2017performance}
Kozhirbayev, Z. \& Sinnott, R. A performance comparison of container-based technologies for the cloud.  {\em Future Generation Computer Systems}. 	\textbf{68} pp. 175--182 (2017).

\bibitem{amaral2015performance}
Amaral, M., Polo, J., Carrera, D., Mohomed, I., Unuvar, M. \& Steinder, M. Performance evaluation of microservices architectures using containers. {\em 2015 IEEE 14th International Symposium on Network Computing and Applications} pp. 27--34 (2015).

\end{thebibliography}




% that's all folks
\end{document}


